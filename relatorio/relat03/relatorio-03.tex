\documentclass[a4paper, 12pt]{article}

\usepackage[portuges]{babel}
\usepackage[utf8]{inputenc}
\usepackage{amsmath}
\usepackage{indentfirst}
\usepackage{graphicx}
\usepackage{multicol,lipsum}

\begin{document}
	%\maketitle
	
	\begin{titlepage}
		\begin{center}
			
			%\begin{figure}[!ht]
			%\centering
			%\includegraphics[width=2cm]{c:/ufba.jpg}
			%\end{figure}
			
			\Huge{Universidade de São Paulo/Instituto de Astronomia, Geofísica e Ciências Atmosféricas}\\
%			\large{Departamento}\\ 
%			\large{Programa}\\ 
			\vspace{15pt}
			\vspace{95pt}
			\textbf{\LARGE{Atividade Prática 3 - TOPCAT}}\\
			%\title{{\large{Título}}}
			\vspace{3,5cm}
		\end{center}
		
		\begin{flushleft}
			\begin{tabbing}
				Aluno: Douglas Ribas de Mattos\\
				nºUSP: 11010930\\
%				Professor orientador: \\
%				Professor co-orientador: \\
			\end{tabbing}
		\end{flushleft}
		\vspace{1cm}
		
		\begin{center}
			\vspace{\fill}
			Maio\\
			2023
		\end{center}
	\end{titlepage}
	%%%%%%%%%%%%%%%%%%%%%%%%%%%%%%%%%%%%%%%%%%%%%%%%%%%%%%%%%%%
	
	% % % % % % % % %FOLHA DE ROSTO % % % % % % % % % %
	
	\begin{titlepage}
		\begin{center}
			
			%\begin{figure}[!ht]
			%\centering
			%\includegraphics[width=2cm]{c:/ufba.jpg}
			%\end{figure}
			
			\Huge{Universidade de São Paulo}\\
			\large{Instituto de Astronomia, Geofísica e Ciências Atmosféricas}\\ 
%			\large{Programa}\\ 
			\vspace{15pt}
			
			\vspace{85pt}
			
			\textbf{\LARGE{Relatório}}
			\title{\large{Atividade Prática 3 TOPCAT}}
			%	\large{Modelo\\
				%   		Validação do modelo clássico}
			
		\end{center}
		\vspace{1,5cm}
		
		\begin{flushright}
			
			\begin{list}{}{
					\setlength{\leftmargin}{4.5cm}
					\setlength{\rightmargin}{0cm}
					\setlength{\labelwidth}{0pt}
					\setlength{\labelsep}{\leftmargin}}
				
				\item Determinação das magnitudes absolutas das estrelas do aglomerado aberto NGC-7092 (ou M39) utilizando o TOPCAT.
				
				\begin{list}{}{
						\setlength{\leftmargin}{0cm}
						\setlength{\rightmargin}{0cm}
						\setlength{\labelwidth}{0pt}
						\setlength{\labelsep}{\leftmargin}}
					
					\item Aluno: Douglas Ribas de Mattos\
%					\item Professor orientador: \
%					\item Professor co-orientador: \
					
				\end{list}
			\end{list}
		\end{flushright}
		\vspace{1cm}
		\begin{center}
			\vspace{\fill}
			Maio\\
			2023
		\end{center}
	\end{titlepage}
	\newpage
	% % % % % % % % % % % % % % % % % % % % % % % % % %
	\newpage
	\tableofcontents
	\thispagestyle{empty}
	
	\newpage
	\pagenumbering{arabic}
	% % % % % % % % % % % % % % % % % % % % % % % % % % %
	\section{Resumo}
	\newpage
	\section{Apresentação}
	\newpage
	
	\section{Descrição de atividades}
	
	\newpage
	\section{Análise dos Resultados}
	\newpage
	\section{Trabalhos Futuros}
	\newpage
	
	\addcontentsline{toc}{section}{Bibliografia}
	\section*{Bibliografia}
	\footnotesize{
		
		\noindent AGUIRRE, L. A. Introdução à Identificação de Sistemas, Técnicas Lineares e Não lineares Aplicadas a Sistemas Reais. Belo Horizonte, Brasil, EDUFMG. 2004.\\
		
	}
	\newpage
	\addcontentsline{toc}{section}{Anexo}
	\section*{Anexo}
\end{document}